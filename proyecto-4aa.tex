\documentclass[12pt]{article}
\usepackage[utf8]{inputenc}
\usepackage[spanish]{babel}
\usepackage{geometry}
\usepackage{tikz}
\usetikzlibrary{positioning, arrows.meta}
\usepackage{graphicx}
\usepackage{amsmath}
\usepackage{xcolor}
\geometry{a4paper, margin=2.5cm}
\title{Proyecto 3: Hamilton, Euler y Grafos, Parte I}
\author{Miembros del Grupo:\\Ricardo Castro\\Juan Carlos Valverde\\~\\Curso: Analisis de Algoritmos\\~\\Semestres: II 2025}
\date{\today}

\begin{document}

\maketitle

\thispagestyle{empty}

\newpage

\section{William Rowan Hamilton}

William Rowan Hamilton (1805-1865) fue un matemático, físico y astrónomo irlandés. Hizo importantes contribuciones al desarrollo del álgebra, la óptica y la mecánica. Es especialmente conocido por su trabajo en el álgebra de cuaterniones y por el problema del ciclo hamiltoniano, que lleva su nombre. El problema consiste en encontrar un ciclo en un grafo que visite cada vértice exactamente una vez.

\section{Ciclos y Rutas Hamiltonianas}

Un \textbf{ciclo hamiltoniano} es un ciclo en un grafo que visita cada vértice exactamente una vez y regresa al vértice inicial. Una \textbf{ruta hamiltoniana} es un camino simple que visita cada vértice exactamente una vez, pero no necesariamente regresa al punto de partida.

El problema de determinar si un grafo tiene un ciclo o ruta hamiltoniana es un problema NP-completo. En este proyecto, se utiliza un algoritmo de backtracking para determinar si existe al menos un ciclo o ruta hamiltoniana, aunque no se encuentra la solución específica.

\section{Leonhard Euler}

Leonhard Euler (1707-1783) fue un matemático y físico suizo considerado uno de los matemáticos más prolíficos de la historia. Realizó importantes descubrimientos en cálculo, teoría de grafos, teoría de números y muchas otras áreas. El problema de los puentes de Königsberg, que resolvió, es considerado el origen de la teoría de grafos.

\section{Ciclos y Rutas Eulerianas}

Un \textbf{ciclo euleriano} es un ciclo que recorre cada arista del grafo exactamente una vez y regresa al vértice inicial. Un \textbf{camino euleriano} (o ruta euleriana) es un camino que recorre cada arista exactamente una vez, pero no necesariamente regresa al punto de partida.

Un grafo es \textbf{euleriano} si tiene un ciclo euleriano. Un grafo es \textbf{semieuleriano} si tiene un camino euleriano pero no un ciclo euleriano.

Para grafos no dirigidos: un grafo es euleriano si y solo si es conexo y todos los vértices tienen grado par. Es semieuleriano si es conexo y tiene exactamente dos vértices de grado impar.

Para grafos dirigidos: un grafo es euleriano si y solo si es fuertemente conexo y cada vértice tiene el mismo grado de entrada que de salida. Es semieuleriano si es conexo y tiene exactamente un vértice con grado de salida mayor que el de entrada en una unidad, y exactamente un vértice con grado de entrada mayor que el de salida en una unidad, y todos los demás tienen grados iguales.

\section{Grafo Original}

\begin{center}
\begin{tikzpicture}[scale=1.00]
\draw[->, thick] (1.00,0.00) -- (0.00,2.00);
\draw[->, thick] (1.00,0.00) -- (6.00,0.00);
\draw[->, thick] (8.00,0.00) -- (5.00,0.00);
\draw[->, thick] (3.00,0.00) -- (0.00,12.00);
\draw[->, thick] (0.00,7.00) -- (10.00,0.00);
\draw[->, thick] (9.00,0.00) -- (8.00,0.00);
\draw[->, thick] (0.00,11.00) -- (5.00,0.00);
\draw[->, thick] (6.00,0.00) -- (0.00,4.00);
\draw[->, thick] (6.00,0.00) -- (0.00,12.00);
\node[circle, draw=black, fill=blue!30, minimum size=0.8cm, font=\scriptsize] (n0) at (1.00,0.00) {0};
\node[circle, draw=black, fill=yellow!30, minimum size=0.8cm, font=\scriptsize] (n1) at (0.00,2.00) {1};
\node[circle, draw=black, fill=red!30, minimum size=0.8cm, font=\scriptsize] (n2) at (8.00,0.00) {2};
\node[circle, draw=black, fill=yellow!30, minimum size=0.8cm, font=\scriptsize] (n3) at (0.00,4.00) {3};
\node[circle, draw=black, fill=green!30, minimum size=0.8cm, font=\scriptsize] (n4) at (3.00,0.00) {4};
\node[circle, draw=black, fill=blue!30, minimum size=0.8cm, font=\scriptsize] (n5) at (5.00,0.00) {5};
\node[circle, draw=black, fill=green!30, minimum size=0.8cm, font=\scriptsize] (n6) at (0.00,7.00) {6};
\node[circle, draw=black, fill=green!30, minimum size=0.8cm, font=\scriptsize] (n7) at (9.00,0.00) {7};
\node[circle, draw=black, fill=blue!30, minimum size=0.8cm, font=\scriptsize] (n8) at (0.00,12.00) {8};
\node[circle, draw=black, fill=green!30, minimum size=0.8cm, font=\scriptsize] (n9) at (0.00,11.00) {9};
\node[circle, draw=black, fill=yellow!30, minimum size=0.8cm, font=\scriptsize] (n10) at (10.00,0.00) {10};
\node[circle, draw=black, fill=yellow!30, minimum size=0.8cm, font=\scriptsize] (n11) at (6.00,0.00) {11};
\end{tikzpicture}
\end{center}

\subsection{Leyenda de Colores}

\begin{itemize}
\item \fcolorbox{black}{blue!30}{\rule{0.5cm}{0.5cm}} Grado de entrada par, grado de salida par
\item \fcolorbox{black}{green!30}{\rule{0.5cm}{0.5cm}} Grado de entrada par, grado de salida impar
\item \fcolorbox{black}{yellow!30}{\rule{0.5cm}{0.5cm}} Grado de entrada impar, grado de salida par
\item \fcolorbox{black}{red!30}{\rule{0.5cm}{0.5cm}} Grado de entrada impar, grado de salida impar
\end{itemize}

\section{Propiedades del Grafo}

\subsection{Ciclos y Rutas Hamiltonianas}

El grafo \textbf{no contiene un ciclo hamiltoniano}. No existe un ciclo que visite cada vértice exactamente una vez.

El grafo \textbf{no contiene una ruta hamiltoniana}. No existe un camino que visite cada vértice exactamente una vez.

\subsection{Propiedades Eulerianas}

El grafo \textbf{no es euleriano ni semieuleriano}. No existe un ciclo ni un camino que recorra cada arista exactamente una vez.

\end{document}
